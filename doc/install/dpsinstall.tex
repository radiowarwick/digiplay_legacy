\documentclass[a4paper,12pt]{report}
%\documentclass{article}
\usepackage{times,amsmath,amsthm,amssymb,graphicx,color,listings}
\usepackage[left=3cm,right=2cm,top=2.5cm,bottom=3.5cm]{geometry}
%\usepackage[left=3.5cm,right=2.5cm,top=3.0cm,bottom=4.0cm]{geometry}

\definecolor{Brown}{cmyk}{0,0.81,1,0.60}
\definecolor{OliveGreen}{cmyk}{0.64,0,0.95,0.40}
\definecolor{CadetBlue}{cmyk}{0.52,0.57,0.23,0}
\definecolor{Yellow}{rgb}{0.8,0.8,0}
\definecolor{Black}{rgb}{0,0,0}
\definecolor{Red}{rgb}{0.8,0,0}
\lstset{language=bash}
\lstset{frame=ltrb,framesep=5pt,basicstyle=\small,
    keywordstyle=[1]\ttfamily\color{Black}\bfseries,
    identifierstyle=\ttfamily\color{Black},
    commentstyle=\color{CadetBlue},
    stringstyle=\color{Red}\ttfamily,
    showstringspaces=false
}
\numberwithin{equation}{section}
\setcounter{tocdepth}{3}
\setcounter{secnumdepth}{4}
\bibliographystyle{plain}

\begin{document}
\title{RaW Digital Playout System}
\author{Installation Manual}
\maketitle

\chapter{Introduction}
The RaW Digital Playout System comprises several software applications and a suite of management tools, all of which operate using a central database. This document describes an installation procedure to suite the majority of system configurations.

\chapter{Choosing a System Configuration}
Before beginning, it is advisable to first plan how the Digital Playout System will be used. Things to consider:
\begin{itemize}
\item How many studios will the system support?
\item How much audio will the system store?
\item How many users will the system support?
\end{itemize}

\section{Audio Storage}
Audio is stored in raw PCM format (and soon using FLAC), to provide the highest quality broadcast. Raw PCM audio uses 10.56MB/min, so 42.2MB will be required for an average four minute track. On a consumer 200GB hard disk, this equates to just under 5000 tracks. Audio can be written to any mounted local or remote linux file system. Multiple storage \emph{archives} are supported allowing, for instance, multiple disk arrays on multiple servers to be used thus permitting indefinite expansion of the system. This should be taken into consideration when planning the installation. Potential storage configurations are:
\begin{enumerate}
\item Local storage on studio machine.
\item Remote storage on an existing generic file server.
\item Dedicated remote storage on multiple NAS or dedicated playout servers.
\end{enumerate}
The first two options are suitable for small installations with one or two studios. The third option is appropriate for larger installations with many studios or large storage requirements.
\textbf{Note:} It is advisable to use data redundancy techniques, such as RAID1 or RAID5 to reduce the likelihood of data loss. Ideally the audio data should be regularly mirrored to a second off-site server, or backed up to another medium.

\section{Database}
The central database may be located on any host, but must be accessible by every playout system. If you have a dedicated server for audio storage, it is advisable to install the database on that machine. For single or dual studio configurations, the database may be installed on one of the studio machines without significant performance penalties.

\section{Studio Software}
There are two components to the studio software: a management interface, and a playout interface. These may either be run on separate physical systems, or together on the one system using two X servers. A separate physical display is required for each application.

\chapter{Quick Installation}
Installation consists of the following tasks. We will assume the use of a Debian based distribution, but other distributions will follow a similar procedure.
\begin{enumerate}
\item Preliminaries: download and unpacking the software.
\item Install and configure the backend database.
\item Compile and install the management tools and set up an audio store.
\item Compile and install the audio acquisition software.
\item Compile and install the studio software.
\end{enumerate}

\section{Preliminaries}
Download the latest version of digiplay onto every digital playout computer. This guide will assume it has been downloaded to \texttt{/usr/local/src}. As root,
\begin{lstlisting}
cd /usr/local/src
tar -zxvf digiplay-<release>.tar.gz
cd digiplay
\end{lstlisting}
The \texttt{INSTALL} file contains a more detailed description of the archive contents, including a full list of package dependancies.

\section{Install and Configure backend database}
This must be installed on one machine only. The database uses the PostgreSQL database engine.
\begin{enumerate}
\item Install the database server (it is recommended to use PostgreSQL 8.0 or later).
\begin{lstlisting}
apt-get update
apt-get install postgresql
\end{lstlisting}
This will most likely install additional dependancies.
\item Install the \texttt{/etc/digiplay.conf} configuration file.
\begin{lstlisting}
cp doc/digiplay.conf /etc/
\end{lstlisting}
This should be correct for the default installation on the machine running the database server. The database and users specified will be created in the next step.
\item Run the database install script: \texttt{dpsinstall}
\begin{lstlisting}
cd /usr/local/src/digiplay/db
./dpsinstall
\end{lstlisting}
This will parse the configuration file and create the database and required database users.
\end{enumerate}

\section{Compile and Install Management Tools}
The management tools perform administrative functions on the system. They are generally installed on the host serving the database, although this is not mandatory.
\begin{enumerate}
\item Install dependancies
\begin{lstlisting}
apt-get update
apt-get install libreadline-dev libpqxx-dev libpthread-dev \
	make g++
\end{lstlisting}
This may install additional packages required by the above.
\item Compile and install the tools. By default applications are installed in \texttt{/usr/local/bin} and libraries in \texttt{/usr/local/lib}, but this may be altered by editing the \texttt{Makefile} in the root of the digiplay package.
\begin{lstlisting}
cd /usr/local/src/digiplay
make tools
make install
\end{lstlisting}
\item Create an audio archive. You may specify a descriptive name for each audio archive. If this audio archive will not be located on the database host, then it must be mounted over NFS on all digital playout hosts.\par
\textbf{NOTE:} The mount point MUST be consistently named across all hosts.\par
You may list the available audio archives using the same tool.
\begin{lstlisting}
dpsarchive -A "Music1" -l "/mnt/audio1" \
	-r "server:/export/audio1"
dpsarchive -L
\end{lstlisting}
\end{enumerate}

\section{Install the Audio Acquisition software}
This is used to transfer music from CD's onto the system. It simply requires a machine with a CDROM drive.
\begin{enumerate}
\item Install dependancies.
\begin{lstlisting}
apt-get install libreadline-dev cdparanoia
\end{lstlisting}
\item Compile and install the software.
\begin{lstlisting}
make playin
make install
\end{lstlisting}
\end{enumerate}

\section{Install the Studio Software}
There are two components to the studio software. 
\subsection{Studio Management}
\begin{enumerate}
\item Install dependancies
\begin{lstlisting}
apt-get update
apt-get install libpqxx-dev libpthread-dev \
	libldap2-dev libqt3-mt-dev
\end{lstlisting}
\item Compile and install the \texttt{studio\_manage} application.
\begin{lstlisting}
make studio_manage install
\end{lstlisting}
\item Install and modify the configuration file.
\begin{lstlisting}
cp doc/digiplay.conf /etc/
\end{lstlisting}
The \texttt{digiplay.conf} file should now be edited to correctly identify the central database set up earlier by specifying the correct host and port (and database name if you chose a different name).
\end{enumerate}
\subsection{Studio Playout}
\begin{enumerate}
\item Install dependancies
\begin{lstlisting}
apt-get update
apt-get install libpqxx-dev libpthread-dev libqt3-mt-dev
\end{lstlisting}
\item Compile and install the \texttt{studio\_play} application.
\begin{lstlisting}
make studio_play install
\end{lstlisting}
\item Install and modify the configuration as done previously.
\end{enumerate}

\end{document}
